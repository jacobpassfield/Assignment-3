\documentclass{article}

\usepackage[utf8]{inputenc}

\usepackage{helvet}
\renewcommand{\familydefault}{\sfdefault}

\usepackage[left=2.5cm, right=2.5cm, top=2.5cm]{geometry}

\usepackage{amsrefs}

\title{\huge{\bf{A preliminary report on data analysis in fisheries science}}}
\author{Jacob Passfield}
\date{Autumn Term 2020}

\begin{document}

\maketitle

\section{Introduction}

An initial Google search defines a fishery to be the industry of rearing and harvesting fish, and other aquatic life. With fish accounting for approximately seventeen percent of global protein consumption \cite{YouTube2017}, it is no surprise that fisheries contribute one hundred billion dollars to the global economy, supporting roughly 260 million jobs worldwide \cite{Big data}. Thus fisheries serve as a significant ground for economic growth and development in many countries. Beyond economic advantages however, the pressing issues of climate change make analysis in fisheries science paramount in predicting the future of marine populations and, by doing so, protecting an important food source. 

To do this mathematical models on fish population dynamics must be taken and the model parameters estimated from fisheries data. In this project I will use R to accompany my developing knowledge of data analysis techniques and mathematical models with their respective parameters. In this report however, I explore the effects of global warming on marine ecosystems to search for  possible areas to concentrate my project on. Then I introduce an example of a mathematical model called \textit{the mixed effect model} which is likely to be used in my project analysis. 

\section{The effects of global warming on marine ecosystems}

To begin with an ecosystem is the combination of living organisms and their environment in which energy flows through a biological network, or namely, a food web. An ecosystem can vary in size as long as it is comprised of the following four parts: non-living things, like rocks or soil; producers, living organisms that can make their own food, like seaweed or flowers; consumers, living organisms that either eat producers or other bling organisms, like fish or humans; and pathogens that survive on decaying matter, like bacteria or fungus. Therefore a marine ecosystem relates to aquatic biological networks. These networks are more complex and the organic resources more evolved, when compared to other ecosystems. This leads to ocean organisms being more sensitive to changes in the environment, so in turn, they are effected by global warming. 

Numerous articles cite that rising temperatures due to warming affect fish body size, with a prediction that body size will decrease between fourteen and twenty-four percent by 2050. One study found that for every increase in water temperature by one degree Celsius, the size of fish decreased by twenty to thirty percent. 

Fishes are ectotherms (an animal that cannot regulate their own body temperature); when in warmer waters, their metabolism increases and so they require more oxygen to support their bodily functions. This is one result of anthropogenic emissions of carbon dioxide. To think carbon dioxide is absorbed by our oceans and so it is clear that higher emissions contribute to a higher concentration of it in our oceans. Subsequently, as the demand for more oxygen is not available, this leads to fish not getting bigger simply because they cannot support themselves at bigger sizes. One implication of this would be a higher morality rate in smaller species as their size would lead them at a higher risk of predation. 

This does not apply to all fish species however, a study found that smaller-sized fish get smaller, whilst larger fish get larger in warmer waters. It is therefore likely that predatory species will increase with warming, whilst small prey will decrease. Although it is not extensively researched what the consequences of changes in fish body sizes are, it is evident that these alterations will have an impact on the food webs within marine ecosystems.

It has been observed that the diversity of species contributes positively to the stability and functioning of an ecosystem but, due to warming, marine life are changing locations in order to remain in environmental conditions they can tolerate. This leads to the disruption of known interactions between species, necessary for a stable ecosystem, and leads to new interactions being developed. Consequently, the removal of producers or consumers in certain locations will again lead to drastic changes in marine food webs. This is expected to effect entire biological networks in marine ecosystems, from top to bottom and vice versa, and is expected to continue into the the foreseeable future regardless of whether emissions are reduced.

Both consequences to warming will fundamentally alter the biological networks within marine ecosystems. It will also have an impact on fishing; catches will be of less value, with smaller or less or even no fish of a particular species to catch. Furthermore, removing fishes that can adapt to temperature change could reduce species ability to adapt to warming. Therefore, the effects of global warming on marine ecosystems will affect everything, from the composition of fish communities to those who rely on the industry as income and those who need feeding. It is therefore paramount to do something about anthropogenic contribution to warming.  

\section{Data analysis}

Data analysis in fisheries science is critical for the management of fisheries on economic, social, political and, as we have seen, biological grounds. A few examples of the use of fisheries data include: food security; describing the impact of fisheries on economies; identifying exploitation and understanding the effects of climate change. The data is usually collected using standard surveys and is readily available on many websites, notably: the UK's government website or Cefas.

Once the data is collected a thorough exploration of the data will bring its quality to light, which in turn will decide the most useful modelling method to describe or predict hypotheses. In this stage an errors, like outliers, or inconsistencies, like missing values, are identified and corrected. Standard ways to do this involve making box plots, a violin plot or a quantile-quantile plot, for example. R and RStudio are two powerful languages to introduce and aid data exploration, from downloading large sets of data to transforming and visualising it. Two useful guides into this programme and how R can used in fishery science are ‘R for Data Science’ and An Introduction to 'Basic Fisheries Analysis with R’. 

These handbooks also introduce modelling with R, which will be important for interpreting data in order to make appropriate predictions about the effects of warming on marine ecosystems. Modelling is followed after the data is prepped and in this stage a regression is usually implemented. Basic regression models are useful is making predictions only when there is sufficient data and the data is linear. Thus other modelling approaches must be used and are expected to be used in my analysis. Generalised linear models and mixed effect models are a few example of this with the latter being explored later on in this report.

In the studies researched above, the articles all follow a similar structure. They include an abstract, followed by introducing why their research is necessary and relevant. After that they delve into the results of their data analysis and write about what they discovered. Then they discuss what the results mean in the wider context, for example what the possible implications of altering fish body sizes has on food web structures and thus marine ecosystems. At the end of their reports they explain the methods of their data analysis: where the data was found; how it was explored and filtered and then what models were used. In the original article I explored, they used mixed effect models, which is what I then began to research. 

\section{Mixed effect models}

Ecological data analysis can be hard to deal with due to the presence of clustered data. For example certain fish are clustered under different fisheries or grouped by population or species. These different grouping factors may lead to fitting a model with many parameters which decreases the sample size and can leave data entries being not fully independent. As a result, mixed effects models were introduced to develop simple linear regression models and avoid these problems. 

\subsection{The simple linear regression model}

Equation.

List.

Traditionally we take a sample and use it to obtain estimates for the population parameters alpha and beta. This is safe to following these assumptions: normality, homogeneity, a fixed X, independent and using the correct model. Understanding what these assumptions mean is not important at this stage because data cannot be appropriately modelled by linear regression in ecology. Due to the clustered nature of ecological data it is difficult to verify these assumptions so this model is often rejected.

Another reason why this model is often rejected can be demonstrated by the following example. Say we wanted to discover whether temperature affects fish body size, just like in … and so we collect data from nine fisheries and we take five samples from each. Since we take five samples from each fishery, it is expected that the body length values are more related to each other than to the body length values from samples at other fisheries. The model above does not consider this relatedness, and so we lead into mixed effect models. 

\subsection{The linear mixed effect model}




Where
Yi is the (ni x 1) response vector
Xi is the (ni x p) design matrix for fixed effects
Beta is the (p x 1) regression coefficients for fixed effects
Zi is the (ni x q) design matrix for random effects
bi is the (qx1) vector of random effects
Ri is the (ni x 1) vector of undiscovered information.

We assume: 
bi ~ N(0,D)
ei ~ N(0, Ei)
b1….bn,  e1…en independent.

For example, in the study … yi refers to the mean body length of a species. For the fixed effects term: xi regard the mean annual sea surface temperature where the beta is the species-specific correlation to be estimated. The random effects term consider the random effects arisen from spatial variation (think…) the temporal variation across the years the data was collected and other random effects resulting from the site where the data was collected, the weather or from the data collector themselves. Because of the random effects term, this article was able to distinguish whether there was a relationship between sea surface temperature and mean body length because the random effects term took the relatedness between time, space and other errors into account. This model allowed for meaningful results in order to make reliable and accurate predictions about the affect of global warming on fish body sizes. Therefore it is expected that mixed effects modelling will be used in my analysis, too.

\section{Conclusion}

The studies references tend to serve as a warning to the impacts of warming of marine ecosystem by analysing either the changes in fish body size or the spatial trends or even both in different species and the same species in different locations. They conclude that it is paramount to consider the implications of these effects, meaning the extent of these changes remain relatively unexplored. For example, predicting how food webs may change or, more specifically, how body size may change functional role of specific species.

With this in mind I will focus my project on how global warming affects marine ecosystems, perhaps looking at temp size using different data, or spatial trends using the same data. the affects temperature has on body size of fish or spatial variation, perhaps for data not used in these examples. I predict mixed modelling will be used to some extent in my analysis and R for this analysis. 

\section*{References}
\begin{biblist}

\bib{YouTube2017}{article}{
    author={Thai Union Group},
    title={What is a Fishery?},
    date={2017},
    eprint={https://www.youtube.com/watch?v=I-DikSs4kRs}}    

\bib{Big data}{article}{
    author={Luis Diaz},
    title={Big data: the future of sustainable fisheries},
    date={2020},
    eprint={https://fisheries.groupcls.com/big-data-the-future-of-sustainable-fisheries/}}

\end{biblist}

\end{document}
