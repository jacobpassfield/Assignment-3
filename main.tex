\documentclass{article}

\usepackage[utf8]{inputenc}

\usepackage{helvet}
\renewcommand{\familydefault}{\sfdefault}

\usepackage[left=2.5cm, right=2.5cm, top=2.5cm]{geometry}

\usepackage{amsrefs}

\usepackage{amsmath}
\usepackage{bm}

\title{\huge{\bf{A preliminary report on data analysis in fisheries science}}}
\author{Jacob Passfield}
\date{Autumn Term 2020}

\begin{document}

\maketitle

\section{Introduction}

An initial Google search defines a fishery to be the industry of rearing and harvesting fish, and other aquatic life. With fish accounting for approximately seventeen percent of global protein consumption \cite{1}, it is no surprise that fisheries contribute one hundred billion dollars to the global economy, supporting roughly two-hundred and sixty million jobs worldwide \cite{2}. Thus fisheries serve as a significant ground for economic growth and development in many countries. Beyond economic advantages however, the pressing issues of climate change make analysis in fisheries science paramount in predicting the future of marine populations and, by doing so, protecting an important food source. 

Mathematical models on fish population dynamics must be taken to do this, and  model parameters estimated from the collect fisheries data. In the project I will use R to accompany my developing knowledge of data analysis techniques and mathematical models with their respective parameters. In this report however, I explore the effects of global warming on marine ecosystems to search for possible areas to concentrate my project on. Then, I briefly talk about the process of data analysis before I give an example of a mathematical model. The model is called \textit{the mixed effects model} which is likely to be used in my project analysis.

\section{The effects of global warming on marine ecosystems}

To begin with, an ecosystem is the combination of living organisms and their environment in which energy flows through a biological network, or a food web. An ecosystem can vary in size as long as it is comprised of the following four parts: non-living things, like rocks or soil; producers (living organisms that can make their own food) like seaweed or flowers; consumers (living organisms that either eat producers or other living organisms) like fish or humans; and pathogens that survive on decaying matter, like bacteria or fungus \cite{3}. 

Marine ecosystems concern aquatic biological networks. These networks are more complex when compared to other ecosystems. This means ocean organisms are more sensitive to changes in their environment and so are affected by global warming \cite{3}. For example, higher temperatures have modified the plankton ecosystem and the survival rate of young cod in the North Sea has decreased \cite{4}.

Numerous articles cite that rising temperatures due to warming affect fish body size, with a prediction that body size will decrease between fourteen and twenty-four percent by 2050 \cite{5}. One study found that for every increase in water temperature by one degree Celsius, the size of fish decreased by twenty to thirty percent \cite{6}. 

Fishes are ectotherms, meaning they cannot regulate their own body temperature, so in warmer waters their metabolism increases, and therefore, more oxygen is required to support their bodily functions. This is one result of anthropogenic emissions of carbon dioxide. Carbon dioxide is absorbed by our oceans and so it is clear that increased emissions contribute to a higher concentration of it in our oceans. Because of this the demand of more oxygen is not available and so fish are not getting bigger simply because they cannot support themselves at bigger sizes. An implication of smaller species would be a higher morality rate as their size would leave them at a higher risk of predation, just like the young cod \cite{6}.

This does not apply to all fish species however, a study found that smaller-sized fish get smaller, whilst larger fish get larger in warmer waters. It is therefore likely that predatory species will increase with warming and small prey will decrease. Although it is not extensively researched what the consequences of changes in fish body sizes are, it is evident that these alterations will have an impact on the food webs within marine ecosystems \cite{7}.

Furthermore it has been observed that the diversity of species contribute positively to the stability and functioning of an ecosystem. However due to warming, marine life are changing locations in order to remain in the environmental conditions they can tolerate. This leads to the disruption of known interactions between species, necessary for a stable ecosystem, and leads to new interactions being developed. Consequently, the removal of producers or consumers in certain locations will again lead to drastic changes in marine food webs. This is expected to affect entire biological networks in marine ecosystems, and is expected to continue into the the foreseeable future regardless of whether emissions are reduced \cite{8}.

Both consequences to warming will fundamentally alter the biological networks within marine ecosystems. As a result of this I want my project to concentrate on the effects of global warming on marine ecosystems; leading to the potential analysis of how temperature affects size or the spatial trends within certain species of fish.

\section{Data analysis}

Data analysis in fisheries science is critical for the management of fisheries on economic, social, political and, as we have seen, biological grounds \cite{9}. A few examples of the use of fisheries data include: food security; describing the impact of fisheries on economies; identifying exploitation and understanding the effects of climate change \cite{10}. Data is usually collected using standard surveys and is readily available on many websites, notably: the UK’s government website or \textit{Cefas}.

After data collection, the next stage is data exploration. In this stage any errors, like outliers, or inconsistencies are identified and corrected. Standard ways to do this involve making box plots, violin plots or even a quantile-quantile plot, for example \cite{9}. Once the data is prepared, mathematical models are used as a descriptive or predictive measure. For instance, in my project I may describe the relationship between temperature and fish body size in order to predict the effects of global warming on marine ecosystems. 

Moreover, in the modelling stage a regression is usually implemented. Basic regression models are useful in making predictions only when there is sufficient data and the data is linear. Thus other modelling approaches must be considered and are expected to be used in my analysis. Generalised linear models and mixed effect models \cite{9} are a few examples of this with the latter being explored later on in this report.

To carry out the data analysis in my project, I will be using R. R is a powerful statistical tool, containing a variety of packages, that will allow me to process and manipulate a large amount of data. It will then allow me to use models to make appropriate predictions about the effects of warming on marine ecosystems. Two useful introductory guides are \textit{R for Data Science} and \textit{Basic Fisheries Analysis with R}, \cite{11} and \cite{12} respectively.
 
When looking at how articles use data analysis in their research, I discovered that the layout of their reports all followed a similar structure. A notable article was “Fish body sizes change with temperature but not all species shrink with warming” \cite{7} or, Audzijonyte et al (2020). This article studied the relationship between temperature changes on fish body sizes, an area that I may focus my project on. It was interesting to note that the report started with the usual abstract and introduction but discussed and interpreted the results before explaining how the data analysis was used to get said results. It was in this section however, that I noticed the authors, not only used R, but used a mixed effects model for their mathematical model; which I will now introduce.

\section{Mixed effect models}

Ecological data analysis can be hard to deal with due to the presence of clustered data. For example certain fish are clustered under different fisheries or grouped by population or species. We start by looking at the simple linear regression model and its flaws to lead us to mixed effect models. This sections follows closely to a useful book named \textit{Mixed effects models and extensions in ecology with R} \cite{13} or, Zuur et al (2009).

\subsection{The simple linear regression model}

This simple linear regression model can be perceived as the foundation for all other models in statistical analysis. The bivariate linear regression model is defined by:

\[
Y_i = \alpha + X_i \beta + R_i 
\quad \mbox{ where } \quad
R_i \sim N(0, \sigma^2)
\]

In this equation, $Y_i$ is the explanatory variable and $X_i$ is the response variable. Unrecognised information is captured in the residual term, $R_i$, which is assumed to follow a normal distribution with a mean of zero and variance $\sigma^2$. The parameter $\alpha$ corresponds to the population intercept and the parameter $\beta$ is the slope; both are unknown. Usually $\beta$ describes the relationship between $X$ and $Y$.

Traditionally we take a sample from the population and use it to obtain estimates for the population parameters $\alpha$ and $\beta$. This is safe to do if the following assumptions are met: normality, homogeneity, a fixed X, independent and using the correct model. Understanding what these assumptions mean is not important in this report because ecological data cannot often be appropriately modelled by a linear regression. This is due to the clustered nature of ecological data making it difficult to verify these assumptions and so this model is often rejected.

Another reason why this model is often rejected can be demonstrated by the following example. Say we wanted to discover whether temperature affects fish body size, just the same as in Audzijonyte et al (2020), and so we collect data from nine fisheries and take five samples from each. Since we take five samples from each fishery, it is expected that the body length between fishes are more related to each other than to the body length between fishes from samples at other fisheries. The linear regression model does not consider this relatedness \cite{13} and so we are lead to the mixed effects model. 

\subsection{The linear mixed effects model}

The linear mixed effect model is usually given in vector form and is defined as:

\[
\mathbf{Y_i} =
\mathbf{X_i {\boldsymbol\beta}} + 
\mathbf{Z_i b_i} + 
\mathbf{R_i}
\]

To help understand the terms, I have added the dimensions of the vectors and matrices to give:

\[
\overbrace{\mathbf{Y_i}}^{\mbox{$n_i$ x 1}}
\quad = \quad
\overbrace{\underbrace{\mathbf{X_i}}_{\mbox{$n_i$ x $p$}} \quad
\underbrace{\boldsymbol{\beta}}_{\mbox{$p$ x 1}}}^{\mbox{$n_i$ x 1}} 
\quad + \quad
\overbrace{\underbrace{\mathbf{Z_i}}_{\mbox{$n_i$ x $q$}} \quad
\underbrace{\mathbf{b_i}}_{\mbox{$q$ x 1}}}^{\mbox{$n_i$ x 1}}
\quad + \quad
\overbrace{\mathbf{R_i}}^{\mbox{$n_i$ x 1}}
\]

In this equation \cite{14}, $\mathbf{Y_i}$ is the response vector. We have that $\mathbf{X_i}$ is the design matrix for fixed effects and $\boldsymbol{\beta}$ is the regression coefficient for fixed effects. The new $\mathbf{Z_i}$ component is the design matrix for random effects and $\mathbf{b_i}$ is the vector of random effects. As seen in the linear regression model the undiscovered information is captured in the vector $\mathbf{R_i}$.

In the fixed effects term, $\mathbf{X_i {\boldsymbol\beta}}$, the parameters are fixed (they do not vary) and are expected to effect the response variable; they are the parameters to be estimated in analysis. By comparison, the parameters of the random effects term, $\mathbf{Z_i b_i}$, are random variables \cites{14, 15}. The presence of both effects lead to the name mixed effects models. 

To explain this better I copied the mixed effects model used in Audzijonyte et al (2020), which is as follows: $\mu_i = \beta_0 + \beta_1 x_1 + \alpha_{cell(i)} + \alpha_{year(i)} + \alpha_i$ . This is a little different to the equation given above, for starters, it is not in vector form. The $\mu_i$ refers to the mean body length of a certain fish species. The fixed effects term, $x_i$, regard the mean annual sea surface temperature where the $\beta$ is the correlation within each species that is to be estimated. The random effects terms, the $\alpha$, account for the random effects arisen from spatial (cells), temporal (years) and other variations: the variations that influence the correlation between temperature and body size. The mixed effects model take this influence into account whilst showing a reliable relationship between temperature and fish size. For the data within the study, a mixed effects model allowed for meaningful results for the authors to make accurate predictions about the effect of global warming on fish body sizes. It is because of this accuracy and reliability that mixed effects modelling will likely be used in my analysis, too.

\section{Conclusion}

In conclusion, when researching the effects of global warming on marine ecosystems, I discovered that the studies tend to serve as a warning to its impact by either: analysing the changes in fish body sizes or the changes in their spatial trends. With this in mind, I also plan to focus my project on how marine ecosystems will be affected by warming, perhaps looking at the effects that temperature has on size, or has on spatial trends, using data from different fisheries. I also predict mixed modelling will be used in my data analysis, to some extent, which will all be coded in R.

\section*{References}
\begin{biblist}

\bib{1}{article}{
    author={Thai Union Group},
    title={What is a Fishery?},
    year={2017},
    eprint={https://www.youtube.com/watch?v=I-DikSs4kRs}
}    

\bib{2}{article}{
    author={Diaz, Luis},
    title={Big data: the future of sustainable fisheries},
    year={2020},
    eprint={https://fisheries.groupcls.com/big-data-the-future-of-sustainable-fisheries/}
}

\bib{3}{article}{
    title={Effects of Global Warming on Marine Ecosystems},
    author={Matishov, GG},
    journal={Climate Change, Human Systems and Policy. EOLSS Publishers, Paris, France},
    pages={188--204},
    year={2009}
}

\bib{4}{article}{
    title={Fishing and temperature effects on the size structure of exploited fish stocks},
    author={Chen-Yi Tu and K. Chen and Chih-hao Hsieh},
    journal={Scientific Reports},
    year={2018},
    volume={8}
}

\bib{5}{article}{
    title={Shrinking of fishes exacerbates impacts of global ocean changes on marine ecosystems},
    author={Cheung, William WL and Sarmiento, Jorge L and Dunne, John and Fr{\"o}licher, Thomas L and Lam, Vicky WY and Palomares, ML Deng and Watson, Reg and Pauly, Daniel},
    journal={Nature Climate Change},
    volume={3},
    number={3},
    pages={254--258},
    year={2013},
    publisher={Nature Publishing Group}
}

\bib{6}{article}{
    title={Sound physiological knowledge and principles in modeling shrinking of fishes under climate change},
    author={Pauly, Daniel and Cheung, William WL},
    journal={Global change biology},
    volume={24},
    number={1},
    pages={e15--e26},
    year={2018},
    publisher={Wiley Online Library}
}

\bib{7}{article}{
    title={Fish body sizes change with temperature but not all species shrink with warming},
    author={Audzijonyte, Asta},
    author={Richards, Shane A.},
    author={Stuart-Smith, Rick D.},
    author={others},
    journal={Nat Ecol Evol},
    volume={4},
    year={2020},
    pages={809--814}
}

\bib{8}{article}{
    title={Biodiversity redistribution under climate change: Impacts on ecosystems and human well-being},
    author={Pecl, Gretta T and Ara{\'u}jo, Miguel B and Bell, Johann D and Blanchard, Julia and Bonebrake, Timothy C and Chen, I-Ching and Clark, Timothy D and Colwell, Robert K and Danielsen, Finn and Eveng{\aa}rd, Birgitta and others},
    journal={Science},
    volume={355},
    number={6332},
    year={2017},
    publisher={American Association for the Advancement of Science}
}

\bib{9}{thesis}{
    author = {Fernandes, Jose},
    year = {2011},
    title = {Data analysis advances in marine science for fisheries management: Supervised classification applications}
}

\bib{10}{article}{
    title={Sample-based fishery surveys},
    author={Stamatopoulos, Constantine},
    journal={A technical handbook. FAO Fishery Technical Paper, Rome: FAO},
    volume={425},
    pages={1--3},
    year={2002}
}

\bib{11}{book}{
    author={Grolemund, Garrett},
    author={Wickham, Hadley},
    title={R for Data Science},
    note={avaliable at http://r4ds.had.co.nz/}
}

\bib{12}{book}{
    title={An introduction to basic fisheries analysis with R},
    note={avaliable at https://sfg-ucsb.github.io/fishery-manageR/}
}

\bib{13}{book}{
    author={Zuur, Alain F.},
    author={Ieno, Elena N.},
    author={Walker, Neil J.},
    author={Saveliev, Anatoly A.},
    author={Smith, Graham M.},
    title={Mixed effects models and extensions in ecology with R},
    series={Statistics for Biology and  Health},
    publisher={Springer, New York},
    year={2009},
    pages={17--22, 101--106}
}

\bib{14}{article}{
    author={UCLA: Statistical Consulting Group},
    title={Introduction to linear mixed models},
    eprint={https://stats.idre.ucla.edu/other/mult-pkg/introduction-to-linear-mixed-models/}
}  

\bib{15}{article}{
    author={Hajduk, Gabriela K},
    title={Introduction to linear mixed models},
    eprint={https://ourcodingclub.github.io/tutorials/mixed-models/}
}

\end{biblist}

\end{document}
