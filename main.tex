\documentclass{article}

\usepackage[utf8]{inputenc}

\usepackage{helvet}
\renewcommand{\familydefault}{\sfdefault}

\usepackage[left=2.5cm, right=2.5cm, top=2.5cm]{geometry}

\title{\huge{\bf{A preliminary report on data analysis in fisheries science}}}
\author{Jacob Passfield}
\date{Autumn Term 2020}

\begin{document}

\maketitle

\section{Introduction}

A quick Google search defines a fishery to be the industry of rearing and harvesting fish, and other aquatic life. With fish accounting for approximately seventeen percent of global protein consumption, it is no surprise that fisheries contribute one hundred billion dollars to the global economy, supporting roughly 260 million jobs worldwide. Thus fisheries serve as a significant ground for economic growth and development in many countries. 

Beyond economic advantages however, the pressing issues of climate change make analysis in fisheries science paramount in predicting the future of marine populations and, by doing so, protecting an important food source.

\section{The effects of global warming on marine ecosystems}

\section{Data analysis}

\section{Mixed effect models}

\subsection{Example}

\section{Conclusion}

\bibliographystyle{abbrv}
\bibliography{library}

\end{document}
