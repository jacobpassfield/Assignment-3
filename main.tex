\documentclass{article}

\usepackage[utf8]{inputenc}

\usepackage{helvet}
\renewcommand{\familydefault}{\sfdefault}

\usepackage[left=2.5cm, right=2.5cm, top=2.5cm]{geometry}

\title{\huge{\bf{A preliminary report on data analysis in fisheries science}}}
\author{Jacob Passfield}
\date{Autumn Term 2020}

\begin{document}

\maketitle

\section{Introduction}

A quick Google search defines a fishery to be the industry of rearing and harvesting fish, and other aquatic life. With fish accounting for approximately seventeen percent of global protein consumption, it is no surprise that fisheries contribute one hundred billion dollars to the global economy, supporting roughly 260 million jobs worldwide. Thus fisheries serve as a significant ground for economic growth and development in many countries. 

Beyond economic advantages however, the pressing issues of climate change make analysis in fisheries science paramount in predicting the future of marine populations and, by doing so, protecting an important food source. And with this is mind I began exploring possible areas to focus my project on, starting with how global warming effects marine ecosystems.

\section{The effects of global warming on marine ecosystems}

To begin with an ecosystem is the combination of living organisms and their environment in which energy flows through a biological network, or namely, a food web. An ecosystem can vary in size as long as it is comprised of the following four parts: non-living things, like rocks or soil; producers, living organisms that can make their own food, like seaweed or flowers; consumers, living organisms that either eat producers or other bling organisms, like fish or humans; and pathogens that survive on decaying matter, like bacteria or fungus. Therefore a marine ecosystem relates to aquatic biological networks. These networks are more complex and the organic resources more evolved, when compared to other ecosystems. This leads to ocean organisms being more sensitive to changes in the environment, so in turn, they are effected by global warming. 

Numerous articles cite that rising temperatures due to warming affect fish body size, with a prediction that Boyd size will decrease between fourteen and twenty-four percent by 2050. One study found that for every increase in water temperature by one degree Celsius, the size of fish decreased by twenty to thirty percent. 

Fishes are ectotherms (an animal that cannot regulate their own body temperature); when in warmer waters, their metabolism increases and so they require more oxygen to support their bodily functions. This is one result of anthropogenic emissions of carbon dioxide. To think carbon dioxide is absorbed by our oceans and so it is clear that higher emissions contribute to a higher concentration of it in our oceans. The consequence of this is that the demand for more oxygen is not available, leading to fish not getting bigger simply because they cannot support themselves at bigger sizes. 

\section{Data analysis}

\section{Mixed effect models}

\subsection{Example}

\section{Conclusion}

\bibliographystyle{abbrv}
\bibliography{library}

\end{document}
